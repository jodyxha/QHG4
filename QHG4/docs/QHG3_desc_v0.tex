\documentclass[a4paper,oneside]{article}

\begin{document}.

\subsection*{Description of the Agent-Based Model}
The simulations for this work were performed by our agent-based 
simulation framework {\bf QHG} (``Quantitative Humane Geophylogeny'').
\subsubsection*{Environment}
The agents in {\bf QHG}, live and move on the nodes of a grid (for the simulations described here, 
we used a spherical grid based onthe subdivision of an icosahedron).
Each node of the grid is assigned a set of environmental data pertainiing to its coordinates:
geography (e.g. altitude, sealevels (REF:SEALEVEL), ice (REF:ICE), rivers), climate (temperature, precipitation),
and vegetation (NPP (REF:NPP)).
The carrying capacity of the cell is calulated from its NPP by means of a ramp function, given by the
parameters $K_{min}, K_{max}, NPP_{min}, NPP_{max}$  as follows:
If $NPP_{cur} < NPP_{min}$ then $K_{cur} =  K_{min}$, 
if $NPP_{cur} > NPP_{max}$ then $K_{cur} =  K_{max}$. Otherwise,
\[ K_{cur} = K_{min} + (K_{max}-K_{min})(NPP_{cur} - NPP_{min})/(NPP_{max} - NPP_{min})\]
The environment data is updated every 1'000 steps (years) to reflect changes in climate, ice and sea-level

Seafaring, e.g. island-hopping to Australia, is realized by adding low-probability connections between two 
nodes separated by water

For a full description of the environmental variables cf. table~\ref{tab:env}.

\subsubsection*{Agents}
At every time step an agent can perform different actions such as move, mate and bear offspring, and/or die. 
Movement is strongly influenced by the environment: the direction in which to move is a random choice 
weighted by altitude and carrying capacity of the neighboring nodes, and agents can not move to nodes 
under water or covered with ice.

Birth and death probabilities are calculated with linear functions leading to logistic growth:
\[ 
  p_{birth}  =  b_{0} + (\theta - b_{0})\frac{N}{K},\hspace{0.2in}
  p_{death}  =  d_{0} + (d_{0} - \theta)\frac{N}{K}, 
\]
where $N$ is the number of agents, $K$ is the local carrying capacity and $\theta$ is the turnover.

A description of the agents' configurable attributes can be found in table~\ref{tab:agt}.


\begin{table}[ht!]
  \begin{center}
  \caption{Environmental Variables}
  \label{tab:env}
  \begin{tabular}{cl|l}
    \hline
    \multicolumn{3}{l}{\bf{Geography}}                                 \\
    \hline
    \space & Latitude    &  latitude of node  (radians)                \\
    \space & Longitude   &  longitude of node (radians)                \\ 
    \space & Altitude    &  altitude of node (meters)                  \\
    \space & Ice         &  ice (1) or no ice (0)                      \\
    \space & Water       &  presence of water (e.g. rivers)            \\  
    \space & Coastal     &  close to coast (1) or not (0)              \\ 
    \space & Distances   &  distances to neighboring nodes             \\
    \space & Area        &  area of region associated wit cell         \\
    \hline
    \multicolumn{2}{l}{\bf{Climate}}                                   \\
    \hline
    \space & AnnualMeanTemp &  current annual mean temperature in node \\
    \space & AnnualRainfall &  current annual total rainfall in node   \\
    \hline
    \multicolumn{2}{l}{\bf{Vegetation}}                                \\
    \hline
    \space & NPP            &  Net Primary Production in node          \\
    \hline

  \end{tabular}

  \end{center}
\end{table}



\begin{table}[ht!]
  \begin{center}
  \caption{Agent Attributes}
  \label{tab:agt}
  \begin{tabular}{cl|l}
    \hline
    \multicolumn{3}{l}{\bf{Birth and Death Probabilies}}           \\
    \hline
    \space & Verhulst\_b0                  &  birth probability for $N=0$   \\
    \space & Verhulst\_d0                  &  death probability for $N=0$   \\
    \space & Verhulst\_theta               &  turn-over rate                \\
    \hline 
    \multicolumn{3}{l}{\bf{Fertility}}                                      \\
    \hline
    \space & Fertility\_interbirth         &  interbirth interval (years)   \\
    \space & Fertility\_max\_age           &  maximum fertile age (years)   \\
    \space & Fertility\_min\_age           &  minimum fertile age (years)   \\
    \hline
    \multicolumn{3}{l}{\bf{Old Age Cut-Off}}                                \\
    \hline
    \space & OAD\_max\_age                 &  average life span   (years)   \\
    \space & OAD\_uncertainty              &  variance                      \\
    \hline
    \multicolumn{3}{l}{\bf{NPP-Capacity Conversion}}                                 \\
    \hline
    \space & NPPCap\_K\_max                & $K_{max}$ parameter for ramp function   \\
    \space & NPPCap\_K\_min                & $K_{min}$ parameter for ramp function   \\
    \space & NPPCap\_NPP\_max              & $NPP_{max}$ parameter for ramp function \\
    \space & NPPCap\_NPP\_min              & $NPP_{min}$ parameter for ramp function \\
    \space & NPPCap\_water\_factor         & Capacity increase for presence of water \\
    \space & NPPCap\_coastal\_factor       & Capacity increase for vicinity of coast \\

    \multicolumn{2}{l}{\bf{Navigation}}                                \\
    \hline
    \space & Navigate\_dist0               & reference distance for crossing \\
    \space & Navigate\_prob0               & probability of crossing at reference distance \\
    \space & Navigate\_decay               & factor for exponential decay of probability with distance \\
  \end{tabular}

  \end{center}
\end{table}

\end{document}


